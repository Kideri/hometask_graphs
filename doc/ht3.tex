\documentclass[a4paper,12pt]{article}

\usepackage[utf8]{inputenc}
\usepackage[T2A]{fontenc}
\usepackage[russian]{babel}
\usepackage{fancyhdr}
\usepackage{float}
\usepackage{tikz}

\begin{document}
\pagestyle{fancy}
\fancyhead[L]{Вариант 182}
\fancyhead[R]{Лист {\thepage} из \pageref{LastPage}}
\fancyfoot{}

\noindent
Домашняя работа по дискретной математике \\
<<Алгоритм Франка-Фриша>> \\
Студент: Суркис Антон Игоревич \\
Группа: P3113

\begin{table}[H]
    \centering
    \caption{Исходный граф}
    \begin{tabular}{l*{3}{|cccc}}
        &$e_{1}$&$e_{2}$&$e_{3}$&$e_{4}$&$e_{5}$&$e_{6}$&$e_{7}$&$e_{8}$&$e_{9}$&$e_{10}$&$e_{11}$&$e_{12}$\\
        \hline
        $e_{1}$&0&3&&2&&1&5&4&4&&&1\\
        $e_{2}$&3&0&3&2&&3&&4&&5&4&4\\
        $e_{3}$&&3&0&&&&2&&4&1&4&5\\
        $e_{4}$&2&2&&0&&&1&&&1&2&\\
        \hline
        $e_{5}$&&&&&0&&&4&1&&&\\
        $e_{6}$&1&3&&&&0&&&&&&1\\
        $e_{7}$&5&&2&1&&&0&4&4&&&5\\
        $e_{8}$&4&4&&&4&&4&0&1&2&5&4\\
        \hline
        $e_{9}$&4&&4&&1&&4&1&0&&5&\\
        $e_{10}$&&5&1&1&&&&2&&0&&\\
        $e_{11}$&&4&4&2&&&&5&5&&0&5\\
        $e_{12}$&1&4&5&&&1&5&4&&&5&0\\
    \end{tabular}
\end{table}

\bigskip
\noindent
\begin{minipage}{\textwidth}
Шаг 1\\
\begin{table}[H]
\centering
\caption{Граф на шаге 1}
\begin{tabular}{r*{12}{|c}}
&$e_{1}$&$e_{2}$&$e_{3}$&$e_{4}$&$e_{5}$&$e_{6}$&$e_{7}$&$e_{8}$&$e_{9}$&$e_{10}$&$e_{11}$&$e_{12}$\\
\hline $e_{1}$&&3&&2&&1&5&4&4&&&1\\
\hline $e_{2}$&3&&3&2&&3&&4&&5&4&4\\
\hline $e_{3}$&&3&&&&&2&&4&1&4&5\\
\hline $e_{4}$&2&2&&&&&1&&&1&2&\\
\hline $e_{5}$&&&&&&&&4&1&&&\\
\hline $e_{6}$&1&3&&&&&&&&&&1\\
\hline $e_{7}$&5&&2&1&&&&4&4&&&5\\
\hline $e_{8}$&4&4&&&4&&4&&1&2&5&4\\
\hline $e_{9}$&4&&4&&1&&4&1&&&5&\\
\hline $e_{10}$&&5&1&1&&&&2&&&&\\
\hline $e_{11}$&&4&4&2&&&&5&5&&&5\\
\hline $e_{12}$&1&4&5&&&1&5&4&&&5&\\
\end{tabular}
\end{table}
Ребра сечения $K_{1}$:\\
\mbox{$(e_{1};e_{2})$ длины 3},
\mbox{$(e_{1};e_{4})$ длины 2},
\mbox{$(e_{1};e_{6})$ длины 1},
\mbox{$(e_{1};e_{7})$ длины 5},
\mbox{$(e_{1};e_{8})$ длины 4},
\mbox{$(e_{1};e_{9})$ длины 4},
\mbox{$(e_{1};e_{12})$ длины 1}\\
$Q=5$\\
Слияние вершин:\\
\mbox{$(e_{1},e_{7})$},
\mbox{$(e_{2},e_{10})$},
\mbox{$(e_{3},e_{12})$},
\mbox{$(e_{7},e_{12})$},
\mbox{$(e_{8},e_{11})$},
\mbox{$(e_{9},e_{11})$},
\mbox{$(e_{11},e_{12})$},
\end{minipage}
\bigskip
\noindent
\begin{minipage}{\textwidth}
Шаг 2\\
\begin{table}[H]
\centering
\caption{Граф на шаге 2}
\begin{tabular}{r*{5}{|c}}
&$(e_{1},e_{3},e_{7},e_{8},e_{9},e_{11},e_{12})$&$(e_{2},e_{10})$&$e_{4}$&$e_{5}$&$e_{6}$\\
\hline $(e_{1},e_{3},e_{7},e_{8},e_{9},e_{11},e_{12})$&&4&2&4&1\\
\hline $(e_{2},e_{10})$&4&&2&&3\\
\hline $e_{4}$&2&2&&&\\
\hline $e_{5}$&4&&&&\\
\hline $e_{6}$&1&3&&&\\
\end{tabular}
\end{table}
Ребра сечения $K_{2}$:\\
\mbox{$((e_{1},e_{3},e_{7},e_{8},e_{9},e_{11},e_{12});e_{6})$ длины 1},
\mbox{$((e_{1},e_{3},e_{7},e_{8},e_{9},e_{11},e_{12});e_{5})$ длины 4},
\mbox{$((e_{1},e_{3},e_{7},e_{8},e_{9},e_{11},e_{12});e_{4})$ длины 2},
\mbox{$((e_{1},e_{3},e_{7},e_{8},e_{9},e_{11},e_{12});(e_{2},e_{10}))$ длины 4}\\
$Q=4$\\
Слияние вершин:\\
\mbox{$(e_{1},e_{3},e_{7},e_{8},e_{9},e_{11},e_{12},e_{2},e_{10})$},
\mbox{$(e_{1},e_{3},e_{7},e_{8},e_{9},e_{11},e_{12},e_{5})$},
\end{minipage}
\bigskip
\noindent
\begin{minipage}{\textwidth}
Шаг 3\\
\begin{table}[H]
\centering
\caption{Граф на шаге 3}
\begin{tabular}{r*{3}{|c}}
&$(e_{1},e_{2},e_{3},e_{5},e_{7},e_{8},e_{9},e_{10},e_{11},e_{12})$&$e_{4}$&$e_{6}$\\
\hline $(e_{1},e_{2},e_{3},e_{5},e_{7},e_{8},e_{9},e_{10},e_{11},e_{12})$&&2&3\\
\hline $e_{4}$&2&&\\
\hline $e_{6}$&3&&\\
\end{tabular}
\end{table}
Ребра сечения $K_{3}$:\\
\mbox{$((e_{1},e_{2},e_{3},e_{5},e_{7},e_{8},e_{9},e_{10},e_{11},e_{12});e_{6})$ длины 3},
\mbox{$((e_{1},e_{2},e_{3},e_{5},e_{7},e_{8},e_{9},e_{10},e_{11},e_{12});e_{4})$ длины 2}\\
$Q=3$\\
Слияние вершин:\\
\mbox{$(e_{1},e_{2},e_{3},e_{5},e_{7},e_{8},e_{9},e_{10},e_{11},e_{12},e_{6})$},
\end{minipage}
\bigskip
\noindent
\begin{minipage}{\textwidth}
Шаг 4\\
\begin{table}[H]
\centering
\caption{Граф на шаге 4}
\begin{tabular}{r*{2}{|c}}
&$(e_{1},e_{2},e_{3},e_{5},e_{6},e_{7},e_{8},e_{9},e_{10},e_{11},e_{12})$&$e_{4}$\\
\hline $(e_{1},e_{2},e_{3},e_{5},e_{6},e_{7},e_{8},e_{9},e_{10},e_{11},e_{12})$&&2\\
\hline $e_{4}$&2&\\
\end{tabular}
\end{table}
Ребра сечения $K_{4}$:\\
\mbox{$((e_{1},e_{2},e_{3},e_{5},e_{6},e_{7},e_{8},e_{9},e_{10},e_{11},e_{12});e_{4})$ длины 2}\\
$Q=2$\\
Слияние вершин:\\
\mbox{$(e_{1},e_{2},e_{3},e_{5},e_{6},e_{7},e_{8},e_{9},e_{10},e_{11},e_{12},e_{4})$},
\end{minipage}

\label{LastPage}
\end{document}
